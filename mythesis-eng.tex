%%%===========================================================================================================
%%%@KoreaTech Thesis Template - TeX Base File (English Version) {
	%%	Created, designed, and written by	: Hongkeun Kim
	%%	Affiliation							: School of Mechatronics Engineering
	%%	Published on						: August 7, 2025
	%%	Note								: This template is free of copyright or licensing restrictions 
	%%										  and may be freely modified and used by all students of KoreaTech.
	%%}
%%%===========================================================================================================


\documentclass[bfive,twoside,eng]{koreatechthesis}

% Packages already included: 
%		calc, geometry, kotex, fontenc, indentfirst, setspace, enumitem, hyperref, titletoc, tocloft, caption, footmisc, titlesec
% So you don't need to load these packages.
\usepackage{array,threeparttable}	% Do not remove. These packages are necessary for tables with required format.
\usepackage{amsmath,amsthm,amssymb}
\usepackage{graphicx,pgfplots}
\usepackage{tikz}
\usetikzlibrary{arrows}
\usepackage[font={bf,normalsize}]{subfig}
%\usepackage{showframe}		% Uncommenting this line shows the layout of the paper.



%-- Environments ---------------------------------------------------------------------------------------------

\theoremstyle{plain}
\makeatletter  
\def\@endtheorem{\hfill\ensuremath{\qedsymbol}\endtrivlist\@endpefalse } % Insert `\qed` macro
\makeatother
\newtheorem{thm}{Theorem}[chapter]
\newtheorem{cor}[thm]{Corollary}
\newtheorem{lem}[thm]{Lemma}
\newtheorem{prop}[thm]{Proposition}
\newtheorem{rem}{Remark}[chapter]
\newtheorem{defn}{Definition}[chapter]
\newtheorem{assmpt}{Assumption}[chapter]
\renewenvironment{proof}{\textit{Proof.}}{\hfill\ensuremath{\blacksquare}}

\allowdisplaybreaks		% To allow line breaks of multi-lined equations over the consecutive pages.



%-- Necessary information ------------------------------------------------------------------------------------

\author{Mecha Kim}			% Give your name.
%\authoreng{Mecha Kim}		% Give your English name: This command applies only to the Korean version and should not be used in the English version.
\title{A\,\LaTeX\,Template\,for\,KoreaTech\,Theses}		% Give the title of your thesis. Note that instead of regular spaces, \, is intentionally used to keep the current title on one line.
\subtitle{Including a Short Guide}						% Give the subtitle. If your thesis does not have subtitle, then leave the argument blank, i.e., \subtitle{}.
%\titleeng{\rm A \LaTeX~Template for KoreaTech Theses}	 % Give the English title of your thesis: This command applies only to the Korean version and should not be used in the English version.
\keywords{주제어1, 주제어2, 주제어3, 주제어4, 주제어5}	   % Give Korean keywords.
\keywordseng{keyword 1, keyword 2, keyword 3, keyword 4, keyword 5}		% Give keywords in English.

\degree{Doctor of Philosophy}		% Give the degree: Master of Science or Doctor of Philosophy
\advisor{Kidae Han}					% Give the name of your advisor. 

\degreeyear{2025}
\degreemonth{February}

\department{Department of Mechatronics Engineering}		% Give your department name
\major{Mechatronics Engineering}						% Give your major
\field{Engineering}				% Give appropriate academic field, e.g., Engineering and Management.

\numberofcommitteemembers{5}	% Give the number of committee members including the chair. The number n should be such that 3 <= n <= 5.
\committeemember{1}{A}
\committeemember{2}{B}
\committeemember{3}{C}
\committeemember{4}{D}
\committeemember{5}{E}
% The following is the information to be printed on the spine of your thesis. 
% The font size must not exceed that of \large; this restriction is enforced by the class file.
\spinetitle{A \LaTeX~Template for \\KoreaTech Theses}	% If needed, a line break can be adjusted manually. Do not let the title on the spine go beyond two lines.
\spineauthor{Mecha Kim}									% If needed, a line break can be adjusted manually. Do not let your name on the spine go beyond two lines.
\spinedegreeyear{2\\0\\2\\5}	% Separate each character by a line break.
\spinedegreemonth{1\\2}			% Enter the month as a number for use on the spine.



%-- Document BEGIN -------------------------------------------------------------------------------------------

\begin{document}

%%%%%%%%%%%%%%%%%%%%%%%%%%%%%%%%%%%%%%%%%%%%%%%%%%%%%%%%%%%%%%%%%%%%%%%%%%%%%%%%%%%%%%%%%%%%%%%%%%%%%%%%%%%%%%
%%%%%%%%%%%%%%%%%%%%%%%%%%%%%%%%%%%%%%%%%%%%%%%%%%%%%%%%%%%%%%%%%%%%%%%%%%%%%%%%%%%%%%%%%%%%%%%%%%%%%%%%%%%%%%
%% Front matter
%%%%%%%%%%%%%%%%%%%%%%%%%%%%%%%%%%%%%%%%%%%%%%%%%%%%%%%%%%%%%%%%%%%%%%%%%%%%%%%%%%%%%%%%%%%%%%%%%%%%%%%%%%%%%%
%%%%%%%%%%%%%%%%%%%%%%%%%%%%%%%%%%%%%%%%%%%%%%%%%%%%%%%%%%%%%%%%%%%%%%%%%%%%%%%%%%%%%%%%%%%%%%%%%%%%%%%%%%%%%%

\makefrontmatter



%%%%%%%%%%%%%%%%%%%%%%%%%%%%%%%%%%%%%%%%%%%%%%%%%%%%%%%%%%%%%%%%%%%%%%%%%%%%%%%%%%%%%%%%%%%%%%%%%%%%%%%%%%%%%%
%%%%%%%%%%%%%%%%%%%%%%%%%%%%%%%%%%%%%%%%%%%%%%%%%%%%%%%%%%%%%%%%%%%%%%%%%%%%%%%%%%%%%%%%%%%%%%%%%%%%%%%%%%%%%%
%% Acknowledgement in Korean
%%%%%%%%%%%%%%%%%%%%%%%%%%%%%%%%%%%%%%%%%%%%%%%%%%%%%%%%%%%%%%%%%%%%%%%%%%%%%%%%%%%%%%%%%%%%%%%%%%%%%%%%%%%%%%
%%%%%%%%%%%%%%%%%%%%%%%%%%%%%%%%%%%%%%%%%%%%%%%%%%%%%%%%%%%%%%%%%%%%%%%%%%%%%%%%%%%%%%%%%%%%%%%%%%%%%%%%%%%%%%

\begin{acknowledgement}
	Acknowledgments should not exceed one page.
\end{acknowledgement}



%%%%%%%%%%%%%%%%%%%%%%%%%%%%%%%%%%%%%%%%%%%%%%%%%%%%%%%%%%%%%%%%%%%%%%%%%%%%%%%%%%%%%%%%%%%%%%%%%%%%%%%%%%%%%%
%%%%%%%%%%%%%%%%%%%%%%%%%%%%%%%%%%%%%%%%%%%%%%%%%%%%%%%%%%%%%%%%%%%%%%%%%%%%%%%%%%%%%%%%%%%%%%%%%%%%%%%%%%%%%%
%% Korean Abstract
%%%%%%%%%%%%%%%%%%%%%%%%%%%%%%%%%%%%%%%%%%%%%%%%%%%%%%%%%%%%%%%%%%%%%%%%%%%%%%%%%%%%%%%%%%%%%%%%%%%%%%%%%%%%%%
%%%%%%%%%%%%%%%%%%%%%%%%%%%%%%%%%%%%%%%%%%%%%%%%%%%%%%%%%%%%%%%%%%%%%%%%%%%%%%%%%%%%%%%%%%%%%%%%%%%%%%%%%%%%%%

\hypersetup{pageanchor=true}	% Page anchoring in the hyperref package is set to be true so as to get a correct link between the pdf bookmark and its associated page number.
\pagestyle{plain}				% No headers, but footers with page numberings only.
\pagenumbering{roman}
\begin{abstractkor}
\setcounter{page}{1}
	국문요약은 2 page 이내로 작성한다.
	주제어는 preamble에 위치한 \verb|\keywords| 명령어를 이용하여 설정하되, 5단어 이내로 작성한다.
	
	Abstract in Korean should be written in Korean and should not exceed two pages.
	The Korean keywords can be specified using the \verb|\keywords| command in the preamble and are limited to no more than five keywords.
\end{abstractkor}



%%%%%%%%%%%%%%%%%%%%%%%%%%%%%%%%%%%%%%%%%%%%%%%%%%%%%%%%%%%%%%%%%%%%%%%%%%%%%%%%%%%%%%%%%%%%%%%%%%%%%%%%%%%%%%
%%%%%%%%%%%%%%%%%%%%%%%%%%%%%%%%%%%%%%%%%%%%%%%%%%%%%%%%%%%%%%%%%%%%%%%%%%%%%%%%%%%%%%%%%%%%%%%%%%%%%%%%%%%%%%
%% Table of contents, lists of figures and tables
%%%%%%%%%%%%%%%%%%%%%%%%%%%%%%%%%%%%%%%%%%%%%%%%%%%%%%%%%%%%%%%%%%%%%%%%%%%%%%%%%%%%%%%%%%%%%%%%%%%%%%%%%%%%%%
%%%%%%%%%%%%%%%%%%%%%%%%%%%%%%%%%%%%%%%%%%%%%%%%%%%%%%%%%%%%%%%%%%%%%%%%%%%%%%%%%%%%%%%%%%%%%%%%%%%%%%%%%%%%%%
\newpage
\addcontentsline{toc}{chapter}{Table of Contents}
\tableofcontents
\newpage
\addcontentsline{toc}{chapter}{List of Tables}
\listoftables
\newpage
\addcontentsline{toc}{chapter}{List of Figures}
\listoffigures

% \cleardoublepage		% do not remove
\newpage
	
	

%%%%%%%%%%%%%%%%%%%%%%%%%%%%%%%%%%%%%%%%%%%%%%%%%%%%%%%%%%%%%%%%%%%%%%%%%%%%%%%%%%%%%%%%%%%%%%%%%%%%%%%%%%%%%%
%%%%%%%%%%%%%%%%%%%%%%%%%%%%%%%%%%%%%%%%%%%%%%%%%%%%%%%%%%%%%%%%%%%%%%%%%%%%%%%%%%%%%%%%%%%%%%%%%%%%%%%%%%%%%%
%% Main body of the thesis
%%%%%%%%%%%%%%%%%%%%%%%%%%%%%%%%%%%%%%%%%%%%%%%%%%%%%%%%%%%%%%%%%%%%%%%%%%%%%%%%%%%%%%%%%%%%%%%%%%%%%%%%%%%%%%
%%%%%%%%%%%%%%%%%%%%%%%%%%%%%%%%%%%%%%%%%%%%%%%%%%%%%%%%%%%%%%%%%%%%%%%%%%%%%%%%%%%%%%%%%%%%%%%%%%%%%%%%%%%%%%
	
\setcounter{page}{1}
\pagenumbering{arabic}

%-- !TEX base = mythesis-eng.tex
%-- mythesis-eng_Guide.tex BEGIN -----------------------------------------------------------------------------

%%%%%%%%%%%%%%%%%%%%%%%%%%%%%%%%%%%%%%%%%%%%%%%%%%%%%%%%%%%%%%%%%%%%%%%%%%%%%%%%%%%%%%%%%%%%%%%%%%%%%%%%%%%%%%
%%%%%%%%%%%%%%%%%%%%%%%%%%%%%%%%%%%%%%%%%%%%%%%%%%%%%%%%%%%%%%%%%%%%%%%%%%%%%%%%%%%%%%%%%%%%%%%%%%%%%%%%%%%%%%
\chapter{Guidelines for \LaTeX~Thesis Template}	\label{chap:Guide}
%%%%%%%%%%%%%%%%%%%%%%%%%%%%%%%%%%%%%%%%%%%%%%%%%%%%%%%%%%%%%%%%%%%%%%%%%%%%%%%%%%%%%%%%%%%%%%%%%%%%%%%%%%%%%%
%%%%%%%%%%%%%%%%%%%%%%%%%%%%%%%%%%%%%%%%%%%%%%%%%%%%%%%%%%%%%%%%%%%%%%%%%%%%%%%%%%%%%%%%%%%%%%%%%%%%%%%%%%%%%%

\begin{quote}
	{\bf Note:} 
	This document was originally written in Korean and then translated into English by a machine, followed by manual revisions by myself. 
	You are kindly asked to take this into consideration.\\
	This document was compiled using PdfLaTeX in TeXstudio version 4.8.7, with TeX Live 2020 as a \LaTeX~engine. 
	It is therefore recommended to use these or later versions of both the \LaTeX~editor and engine to ensure full compatibility.
\end{quote}

This \LaTeX~thesis template follows the standard thesis submission guidelines of Korea University of Technology and Education \cite{ThesisGuide}.

To compile this \LaTeX~thesis template, it is required to use the class file \verb|koreatechthesis.cls|, which implements the official thesis format of KoreaTech \cite{ThesisTemplate}.
This class file must be located in the same folder as the TeX base file \verb|mythesis-eng.tex|. 
The correct output will be generated when the compilation is performed from the TeX base file.

To provide an example of efficient thesis management, this template organizes each chapter in a separate \verb|*.tex| file using the \verb|\include| command.
For instance, Chapter \ref{chap:Guide} is written in \verb|mythesis-eng_Guide.tex|, allowing for organized and modular chapter-wise file management. 



%%%%%%%%%%%%%%%%%%%%%%%%%%%%%%%%%%%%%%%%%%%%%%%%%%%%%%%%%%%%%%%%%%%%%%%%%%%%%%%%%%%%%%%%%%%%%%%%%%%%%%%%%%%%%%
%%%%%%%%%%%%%%%%%%%%%%%%%%%%%%%%%%%%%%%%%%%%%%%%%%%%%%%%%%%%%%%%%%%%%%%%%%%%%%%%%%%%%%%%%%%%%%%%%%%%%%%%%%%%%%
\section{Document Formatting and documentclass}	\label{sec:Guide_Documentclass}
%%%%%%%%%%%%%%%%%%%%%%%%%%%%%%%%%%%%%%%%%%%%%%%%%%%%%%%%%%%%%%%%%%%%%%%%%%%%%%%%%%%%%%%%%%%%%%%%%%%%%%%%%%%%%%

The TeX base file \verb|mythesis-eng.tex| begins with the \verb|\documentclass{korea| \verb|techthesis}| command.
The class file \verb|koreatechthesis.cls| is designed to accept three categories of options.
In terms of language, it allows the use of either the \verb|kor| or \verb|eng| option.
For example, to write your thesis in English, the command should be written as \verb|\documentclass[eng]{koreatechthesis}|.
The remaining two types of options will be explained in the following subsections.


%%%%%%%%%%%%%%%%%%%%%%%%%%%%%%%%%%%%%%%%%%%%%%%%%%%%%%%%%%%%%%%%%%%%%%%%%%%%%%%%%%%%%%%%%%%%%%%%%%%%%%%%%%%%%%
\subsection{oneside vs. twoside}	\label{sec:Guide_Documentclass_SideOption}
%%%%%%%%%%%%%%%%%%%%%%%%%%%%%%%%%%%%%%%%%%%%%%%%%%%%%%%%%%%%%%%%%%%%%%%%%%%%%%%%%%%%%%%%%%%%%%%%%%%%%%%%%%%%%%

When the \verb|twoside| option is specified in the class file, that is, \verb|\documentclass| \verb|[twoside,eng]{koreatechthesis}|, it follows the conventional typesetting practices used in printed books.
In most commercially published books, chapters and tables of contents typically begin on odd-numbered (right-hand) pages.
Consequently, the even-numbered page preceding a new chapter may sometimes appear as a blank page, depending on the layout of the previous chapter.
The \verb|twoside| option precisely follows this convention. 
If you intend to print your final thesis double-sided, then enable this option.
For clarity, consider Chapter \ref{chap:Guide} in this template --- its preceding page is a blank page under the \verb|twoside| option.
Another examples can be found in the front matter of this template, where all even-numbered pages appear as blank pages.
If you imagine printing this file double-sided, then the reason for this formatting will become clear.

To print single-sided, the \verb|oneside| option may be used. 
In this case, all blank pages that would appear with the \verb|twoside| option are removed.


%%%%%%%%%%%%%%%%%%%%%%%%%%%%%%%%%%%%%%%%%%%%%%%%%%%%%%%%%%%%%%%%%%%%%%%%%%%%%%%%%%%%%%%%%%%%%%%%%%%%%%%%%%%%%%
\subsection{bfive vs. afour}	\label{sec:Guide_Documentclass_PaperOption}
%%%%%%%%%%%%%%%%%%%%%%%%%%%%%%%%%%%%%%%%%%%%%%%%%%%%%%%%%%%%%%%%%%%%%%%%%%%%%%%%%%%%%%%%%%%%%%%%%%%%%%%%%%%%%%

This section explains the \verb|bfive| and \verb|afour| options.
According to the official thesis guidelines \cite{ThesisGuide}, the final version of your thesis must be printed on B5 size paper ($182 \times 257\,{\rm mm}$)%
\footnote{%
	The size of A4 paper appears to be standardized internationally.
	It conforms to the ISO international standard, with dimensions of $210 \times 297\,{\rm mm}$.
	In contrast, the situation is different for B5 paper.
	In Korea, most of the publishing industry seems to follow the Japanese Industrial Standard (JIS) which defines B5 paper as $182 \times 257\,{\rm mm}$.
	Unfortunately, \LaTeX~adheres to the ISO international standard under which B5 paper measures $176 \times 250\,{\rm mm}$.
	This \LaTeX~template is configured to use the JIS standard B5 paper size in accordance with \cite{ThesisGuide}.
}.
To produce it in B5 format, the \verb|bfive| option should be used.
Note however that while generating a PDF with this option presents no issues, printing on physical papers requires a printer that supports B5 paper size and has B5 sheets loaded in its tray.

\begin{figure}[tp]
	\centering%
	\begin{tikzpicture}[>=stealth]
		% bfive
		\draw[fill=black!15,opacity=0,fill opacity=0.5]	(0.21,-0.3) -- (2.94,-0.3) -- (2.94,-4.155) -- (0.21,-4.155) -- cycle;
		\draw	(0.8,-4.23) node[anchor=south] {\small\verb|bfive|};
		
		\draw	(0.21,0.1) -- (0.21,0.4);
		\draw	(2.94,0.1) -- (2.94,0.4);
		\draw[<->]	(0.21,0.25) -- (2.94,0.25);
		\draw	(1.575,0.25)	node[anchor=south]	{{\footnotesize $182\,{\rm mm}$}};
		\draw[<->]	(2.94,0.25) -- (3.15,0.25);
		\draw[<-]	(3.045,0.32) -- (3.3,0.6) -- (4.4,0.6);
		\draw	(3.95,0.6)	node[anchor=south]	{{\footnotesize $14\,{\rm mm}$}};
		
		\draw	(-0.1,-0.3) -- (-0.4,-0.3);
		\draw	(-0.1,-4.155) -- (-0.4,-4.155);
		\draw[<->]	(-0.25,-0.3) -- (-0.25,-4.155);
		\draw	(-0.45,-2.0)	node[rotate=-90]	{{\footnotesize $257\,{\rm mm}$}};
		\draw[<->]	(-0.25,0) -- (-0.25,-0.3);
		\draw[<-]	(-0.32,-0.15) -- (-0.8,0.4) -- (-2.0,0.4);
		\draw	(-1.5,0.4)	node[anchor=south]	{{\footnotesize $20\,{\rm mm}$}};
		
		% afour
		\draw	(0,0) -- (3.15,0) -- (3.15,-4.455) -- (0,-4.455) -- cycle;
		\draw	(2.6,-4.53) node[anchor=south] {\small\verb|afour|};
		
		\draw	(0,0.1) -- (0,0.9);
		\draw	(3.15,0.1) -- (3.15,0.9);
		\draw[<->]	(0,0.75) -- (3.15,0.75);
		\draw	(1.575,0.75)	node[anchor=south]	{{\footnotesize $210\,{\rm mm}$}};
		
		\draw	(-0.1,0) -- (-0.9,0);
		\draw	(-0.1,-4.455) -- (-0.9,-4.455);
		\draw[<->]	(-0.75,0) -- (-0.75,-4.455);
		\draw	(-0.95,-2.3)	node[rotate=-90]	{{\footnotesize $297\,{\rm mm}$}};
		
		% contents
		\node[text width=2.47cm,anchor=north west,align=justify]	at	(0.21,-0.3) {\footnotesize The content you have written appears only within the gray B5-sized box.};
	\end{tikzpicture}
	\caption{Comparison between afour and bfive options}%
	\label{fig:B5vsA4}
\end{figure}

Since most people print on A4 paper, the \verb|afour| option is also provided for convenience.
When the \verb|afour| option is specified in \verb|koreatechthesis.cls|, the output is generated with dimensions that precisely match those of standard A4 paper.
This means it can be printed easily on most commonly available printers.
However, as illustrated in Figure \ref{fig:B5vsA4}, the actual content is confined to the gray B5-sized area centered on the A4 page.
Therefore, even though you switch back and forth between the \verb|bfive| and \verb|afour| options, the layout of your content --- such as equations, figures, and tables --- will remain unchanged.
So, feel free to switch between the two options whenever necessary.

The following is a summary of Sections \ref{sec:Guide_Documentclass_SideOption} and \ref{sec:Guide_Documentclass_PaperOption}.

\begin{thm}	\label{thm:ClassOptions}
	Regarding to this \LaTeX~template, if you intend to print
	\begin{itemize}
		\item	the final version of your thesis for \emph{double-sided} printing, then use\\
				\verb|\documentclass[bfive,twoside,eng]{koreatechthesis}|
		\item 	the final version of your thesis for \emph{single-sided} printing, then use\\
				\verb|\documentclass[bfive,oneside,eng]{koreatechthesis}|
		\item 	an intermediate draft for review purposes, then use\\
				\verb|\documentclass[afour,twoside,eng]{koreatechthesis}|\\
				or \verb|\documentclass[afour,oneside,eng]{koreatechthesis}|
	\end{itemize}	
	Options other than these are not supported.
\end{thm}

\begin{proof}
	The proof of Theorem \ref{thm:ClassOptions} is omitted.
\end{proof}



%%%%%%%%%%%%%%%%%%%%%%%%%%%%%%%%%%%%%%%%%%%%%%%%%%%%%%%%%%%%%%%%%%%%%%%%%%%%%%%%%%%%%%%%%%%%%%%%%%%%%%%%%%%%%%
%%%%%%%%%%%%%%%%%%%%%%%%%%%%%%%%%%%%%%%%%%%%%%%%%%%%%%%%%%%%%%%%%%%%%%%%%%%%%%%%%%%%%%%%%%%%%%%%%%%%%%%%%%%%%%
\section{Miscellaneous}	\label{sec:Guide_Etc}
%%%%%%%%%%%%%%%%%%%%%%%%%%%%%%%%%%%%%%%%%%%%%%%%%%%%%%%%%%%%%%%%%%%%%%%%%%%%%%%%%%%%%%%%%%%%%%%%%%%%%%%%%%%%%%

This section briefly introduces a few notes related to this \LaTeX~template.
First, if your thesis has a subtitle, then you can use the \verb|\subtitle| command in the preamble of the TeX base file \verb|mythesis-eng.tex|.
If it has no subtitle, simply leave the argument of \verb|\subtitle| empty --- that is, write it as \verb|\subtitle{}|.

The approval page, which contains the signatures of the committee members, is designed to require \emph{handwritten names and signatures} from each committee member.
In accordance with the university regulations, the number of committee members, including the chair, must be no fewer than three and no more than five.
Therefore, this template is constructed to allow flexible adjustment of the number of committee members according to the actual composition of the committee.
Indeed, by specifying an appropriate integer as the argument of \verb|\numberofcommitteemembers| in the preamble of the TeX base file \verb|mythesis-eng.tex|, signature fields for that number of committee members (including the chair) will be generated.
A more precise description is given below.
\begin{lem}
	Let $n$ be the integer you specified as the argument of the command \verb|\numberofcommitteemembers|.
	Then, the signature fields for
	\begin{enumerate}[label=(\alph*)]
		\item 	three committee members if $n < 3$,
		\item 	five committee members if $n > 5$, and
		\item 	exactly $n$ committee members otherwise
	\end{enumerate}
	will be generated on the approval page.
\end{lem}

Chapters and sections are structured using the commands \verb|\chapter| and \verb|\section|, respectively.
The section depth can go down to the subsubsection level.
In other words, the hierarchy of chapters and sections supported by this \LaTeX~template is as follows.
\begin{verbatim}
	\chapter
	\section
	\subsection
	\subsubsection
\end{verbatim}
However, according to \cite{ThesisGuide} and the official university template \cite{ThesisTemplate} based on it, titles created using \verb|\subsubsection| are not included in the Table of Contents.
Accordingly, this \LaTeX~template is designed to include only up to the \verb|\subsection| level in the Table of Contents.

In the preamble of \verb|mythesis-eng.tex|, the \verb|\newtheorem| command is used to define new environments for theorems and definitions.
In this setup, the numbering for the environments is based on the chapter level.
If you prefer to number them by section instead, you can simply change the option \verb|chapter| to \verb|section|.

This \LaTeX~template uses the \verb|hyperref| package.
As a result, red and green boxes are generated in the output PDF, and clicking these boxes will correctly navigate to the linked locations.
Of course, these boxes are not printed when the PDF is sent to a physical printer.

% This \LaTeX~template does not enforce a particular style for the bibliography.
% Therefore, you should write the \verb|thebibliography| section in \verb|mythesis-eng.tex| as specified in \cite{ThesisGuide}.

The unofficial template can use a .bib file like \cite{2024RN790combo}.


\include{mythesis-eng_LaTeX}



%%%%%%%%%%%%%%%%%%%%%%%%%%%%%%%%%%%%%%%%%%%%%%%%%%%%%%%%%%%%%%%%%%%%%%%%%%%%%%%%%%%%%%%%%%%%%%%%%%%%%%%%%%%%%%
%%%%%%%%%%%%%%%%%%%%%%%%%%%%%%%%%%%%%%%%%%%%%%%%%%%%%%%%%%%%%%%%%%%%%%%%%%%%%%%%%%%%%%%%%%%%%%%%%%%%%%%%%%%%%%
%% References
%%%%%%%%%%%%%%%%%%%%%%%%%%%%%%%%%%%%%%%%%%%%%%%%%%%%%%%%%%%%%%%%%%%%%%%%%%%%%%%%%%%%%%%%%%%%%%%%%%%%%%%%%%%%%%
%%%%%%%%%%%%%%%%%%%%%%%%%%%%%%%%%%%%%%%%%%%%%%%%%%%%%%%%%%%%%%%%%%%%%%%%%%%%%%%%%%%%%%%%%%%%%%%%%%%%%%%%%%%%%%

\cleardoublepage
\phantomsection		% \phantomsection -> chapter/section anchoring
\addcontentsline{toc}{chapter}{References}
\bibliographystyle{unofficial-koreatech}
\bibliography{bib_sample}




%%%official reference style
% \begin{spacing}{1.2}
% \begin{thebibliography}{99}
% 	\bibitem{ThesisGuide}
% 		Guidelines on Formatting and Submission of Master's and Doctoral Dissertations, {\it Korea University of Technology \& Education}, Revised on Nov. 30, 2023.
		
% 	\bibitem{ThesisTemplate}
% 		English ver. dissertation sample, {\it Korea University of Technology \& Education}, \url{https://www.koreatech.ac.kr/menu.es?mid=a50403020000}
		
% 	\bibitem{Khalil}
% 		H. K. Khalil, {\it Nonlinear Systems (3rd ed.)}, Prentice Hall, 2002.
% \end{thebibliography}
% \end{spacing}



%%%%%%%%%%%%%%%%%%%%%%%%%%%%%%%%%%%%%%%%%%%%%%%%%%%%%%%%%%%%%%%%%%%%%%%%%%%%%%%%%%%%%%%%%%%%%%%%%%%%%%%%%%%%%%
%%%%%%%%%%%%%%%%%%%%%%%%%%%%%%%%%%%%%%%%%%%%%%%%%%%%%%%%%%%%%%%%%%%%%%%%%%%%%%%%%%%%%%%%%%%%%%%%%%%%%%%%%%%%%%
%% Abstract in English
%%%%%%%%%%%%%%%%%%%%%%%%%%%%%%%%%%%%%%%%%%%%%%%%%%%%%%%%%%%%%%%%%%%%%%%%%%%%%%%%%%%%%%%%%%%%%%%%%%%%%%%%%%%%%%
%%%%%%%%%%%%%%%%%%%%%%%%%%%%%%%%%%%%%%%%%%%%%%%%%%%%%%%%%%%%%%%%%%%%%%%%%%%%%%%%%%%%%%%%%%%%%%%%%%%%%%%%%%%%%%

\begin{abstracteng}
	Abstract should be written in English and should not exceed two pages.
	The keywords can be specified using the \verb|\keywordseng| command in the preamble and are limited to no more than five keywords.
\end{abstracteng}



%%%%%%%%%%%%%%%%%%%%%%%%%%%%%%%%%%%%%%%%%%%%%%%%%%%%%%%%%%%%%%%%%%%%%%%%%%%%%%%%%%%%%%%%%%%%%%%%%%%%%%%%%%%%%%
%%%%%%%%%%%%%%%%%%%%%%%%%%%%%%%%%%%%%%%%%%%%%%%%%%%%%%%%%%%%%%%%%%%%%%%%%%%%%%%%%%%%%%%%%%%%%%%%%%%%%%%%%%%%%%
%% Appendix
%%%%%%%%%%%%%%%%%%%%%%%%%%%%%%%%%%%%%%%%%%%%%%%%%%%%%%%%%%%%%%%%%%%%%%%%%%%%%%%%%%%%%%%%%%%%%%%%%%%%%%%%%%%%%%
%%%%%%%%%%%%%%%%%%%%%%%%%%%%%%%%%%%%%%%%%%%%%%%%%%%%%%%%%%%%%%%%%%%%%%%%%%%%%%%%%%%%%%%%%%%%%%%%%%%%%%%%%%%%%%

\begin{appendixformatted}
	\section{Appendix Structure}
	In \LaTeX, appendices are typically treated as chapters. Therefore, if further division within an appendix is needed, then use the \verb|\section| command appropriately.
	The appendix section in this template has also been designed in accordance with \cite{ThesisGuide}.
	If your thesis does not include any appendices, then you may delete or comment out all the appendix-related code in this template.
	
	\section{Numbering}
	All appendix numbering starts with the letter A.
	For example, the numbering of equations appears as follows.
	\begin{align}
		\int_0^{2\pi} \sin x \,dx = 0.
	\end{align}
	Likewise, the numbering of figures is as follows.
	\begin{figure*}[h]
		\centering%
		\includegraphics[width=0.4\textwidth]{example-image-a}
		\caption{A figure example in the Appendix}
	\end{figure*}
	
	This \LaTeX~template guide for KoreaTech theses ends here.
	I hope this guide and template support your smooth and successful thesis submission.
\end{appendixformatted}

\end{document}